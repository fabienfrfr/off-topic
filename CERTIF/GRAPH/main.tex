\documentclass{beamer}

% Encodage et langue
\usepackage[utf8]{inputenc}
\usepackage[T1]{fontenc}
\usepackage[french]{babel}

% Packages utiles
\usepackage{amsmath}     % formules mathématiques
\usepackage{amssymb}
\usepackage{listings}    % code source
\usepackage{xcolor}      % couleurs pour listings

% Thème et couleurs
\usetheme{Madrid}        % thème (choix libre)
\usecolortheme{seahorse} % palette de couleurs (optionnel)

% Configuration du package listings pour le code C#
\lstset{
  language=[Sharp]C,
  basicstyle=\ttfamily\small,
  keywordstyle=\color{blue}\bfseries,
  commentstyle=\color{gray}\itshape,
  stringstyle=\color{red},
  numbers=left,
  numberstyle=\tiny,
  stepnumber=1,
  numbersep=5pt,
  showstringspaces=false,
  breaklines=true,
  frame=single,
  backgroundcolor=\color{gray!10}
}


\begin{document}

\title{Knowledge Management \& Graphs Concepts}
\author{Fabien FURFARO}
\date{\today}
\frame{\titlepage}

% Slide Sommaire général
\begin{frame}
  \frametitle{Sommaire}
  \tableofcontents
\end{frame}

% Partie 1 : Présentation du contexte et du besoin
\section{Présentation du contexte et du besoin}

\subsection{Historique}
\begin{frame}
  \frametitle{Historique}

  La gestion des connaissances (Knowledge Management) a évolué en plusieurs phases :
  \begin{itemize}
    \item Avant 1980 : pratiques informelles, catalogage manuel, échanges oraux.
    \item 1980-1995 : premiers outils informatiques et bases de connaissances structurées.
    \item 1995-2010 : plateformes collaboratives, moteurs de recherche spécialisés.
    \item 2010 et après : NLP/NLU, graph databases, transformers, accent sur gouvernance et modélisation.
  \end{itemize}
\end{frame}

\subsection{Défis actuels et objectifs}
\begin{frame}
  \frametitle{Défis actuels et objectifs}
  
  \begin{itemize}
  \item \textbf{Fragmentation} de l'information et silos de données.
  \item \textbf{Hétérogénéité} des systèmes.
  \item \textbf{Perte du savoir} par turnover ou absence de capitalisation.
  \end{itemize}
  
  Objectif : créer des plateformes « Google-like », capables de retrouver et structurer la connaissance avec un \textbf{KMS} (Knowledge Management System) s’appuyant sur curation humaine, ontologie métier, indexation croisée et technologies avancées (ex : knowledge graph, IA).
\end{frame}

\subsection{Composants d’un KMS moderne}
\begin{frame}
  \frametitle{Composants d’un KMS moderne}
  \begin{itemize}
    \item Modèles formels, ontologies.
    \item Moteurs de règle, systèmes experts.
    \item Knowledge graph, sémantique.
    \item IA (NLP/NLU, entity recognition, retrieval).
    \item Interfaces LLM/RAG, intégration outils métier.
  \end{itemize}
\end{frame}

\subsection{Cas d’usage : Curiosity}
\begin{frame}
  \frametitle{Cas d’usage : Curiosity}
  \begin{itemize}
    \item KMS basé sur un \textbf{Linked Property Graph (LPG)} développé en C\#.
    \item Détection automatique et pondération des entités (NLP).
    \item Recherche par similarité (subgraph embedding, vectorisation texte).
    \item Rapidité sur grands volumes (plusieurs centaines de milliers de docs).
  \end{itemize}
\end{frame}

\subsection{Bénéfices et limites}
\begin{frame}
  \frametitle{Bénéfices et limites}
  \begin{itemize}
    \item Gain de temps, pertinence, efficacité de support client
    \item Sécurité du patrimoine informationnel
    \item Défis persistants de désilotage, capitalisation experte, automatisation des process
  \end{itemize}
\end{frame}

\subsection{Exemple concret : aéronautique}
\begin{frame}
  \frametitle{Exemple concret : aéronautique}
  
  Déploiement progressif chez un avionneur : de 15 000 utilisateurs internes à plus de 100 000 clients externes, grâce à la puissance du knowledge graph pour résoudre des problématiques techniques complexes.
\end{frame}

\subsection{Approche Capgemini « Trust before technology »}
\begin{frame}
  \frametitle{Approche Capgemini « Trust before technology »}
  
  \begin{itemize}
    \item Diagnostic métier et technique approfondi, co-construction humaine + technologique, prototypage rapide permettant d’incarner la valeur ajoutée.
  \end{itemize}
\end{frame}

% Partie 2 : Présentation des compétences nécessaires
\section{Présentation des compétences nécessaires}

\subsection{Théorie et concepts fondamentaux}

\subsubsection{Graphes et connaissance}
\begin{frame}
  \frametitle{Graphes et connaissance}

  \begin{itemize}
    \item \textbf{Graphe orienté ou non orienté}: \( G = (V, E) \) avec \( V \) = nœuds (entités), \( E \) = arêtes (relations).
    \item \textbf{Linked Property Graph (LPG)}: chaque nœud/arête peut avoir des propriétés clé-valeur.
    \item \textbf{Requête sur graphe}: trouver des sous-graphes isomorphes, calcul de chemins, recherche de motifs.
  \end{itemize}
\end{frame}

\subsubsection{Sémantique, ontologies}
\begin{frame}
  \frametitle{Sémantique, ontologies}

  \begin{itemize}
    \item \textbf{Ontologie}: formalise les concepts et relations d’un domaine (ex : OWL, RDF Schema) ; utile pour donner du sens et permettre des inférences au-delà du simple graphe.
  \end{itemize}
\end{frame}

\subsubsection{Vectorisation et recherche sémantique}
\begin{frame}
  \frametitle{Vectorisation et recherche sémantique}

  \begin{itemize}
    \item \textbf{Embeddings}: vectorisation des nœuds/document associés pour calculer la similarité (cosinus, euclidienne) :
    \[
    \cos(\theta) = \frac{\mathbf{v}_1 \cdot \mathbf{v}_2}{\|\mathbf{v}_1\|\|\mathbf{v}_2\|}
    \]
    \item Techniques courantes : Word2Vec, GloVe, transformers (BERT/LLM).
  \end{itemize}
\end{frame}

\subsubsection{IA et NLP}
\begin{frame}
  \frametitle{IA et NLP}

  \begin{itemize}
    \item Extraction d’entités : reconnaissance automatique des entités dans un texte.
    \item Similarité sémantique : utiliser des modèles de langage pour relier de nouveaux cas à l’historique.
  \end{itemize}
\end{frame}

\subsection{Mathématiques et algorithmes de base}

\begin{frame}
  \frametitle{Algorithmes fondamentaux}
  \begin{itemize}
    \item Algorithme de parcours de graphe (DFS, BFS).
    \item Pagerank pour déterminer l’importance d’un nœud.
    \item Recherche de plus courts chemins (Dijkstra, A*).
    \item Recherche de sous-graphes similaires (isomorphisme de sous-graphe, graph embedding).
  \end{itemize}
\end{frame}

\begin{frame}[fragile]
  \frametitle{Exemple d’algorithme : BFS en C\#}
\begin{lstlisting}
Queue<Node> queue = new Queue<Node>();
HashSet<Node> visited = new HashSet<Node>();
queue.Enqueue(startNode);

while (queue.Count > 0)
{
    Node current = queue.Dequeue();
    if (!visited.Contains(current))
    {
        visited.Add(current);
        foreach (Node neighbor in current.Neighbors)
        {
            queue.Enqueue(neighbor);
        }
    }
}
\end{lstlisting}
\end{frame}

\begin{frame}[fragile]
  \frametitle{Exemple de similarité cosinus entre deux embeddings}
\begin{lstlisting}
public double CosineSimilarity(double[] v1, double[] v2)
{
    double dot = 0.0, mag1 = 0.0, mag2 = 0.0;
    for (int i = 0; i < v1.Length; i++)
    {
        dot += v1[i] * v2[i];
        mag1 += v1[i] * v1[i];
        mag2 += v2[i] * v2[i];
    }
    return dot / (Math.Sqrt(mag1) * Math.Sqrt(mag2));
}
\end{lstlisting}
\end{frame}

\begin{frame}
  \frametitle{PageRank}
  Formule de base :
  \[
  PR(u) = \frac{1-d}{N} + d \sum_{v \in B_u} \frac{PR(v)}{L(v)}
  \]
  avec \( d \) = facteur d’amortissement (classiquement 0.85), \( N \) = nombre total de nœuds.
\end{frame}

\subsection{Exemple pratique de requête sur un LPG}

\begin{frame}[fragile]
  \frametitle{Définition d’un nœud et d’une relation en C\#}

\begin{lstlisting}
public class Node
{
    public string Id { get; set; }
    public Dictionary<string, object> Properties { get; set; }
}

public class Relationship
{
    public Node From { get; set; }
    public Node To { get; set; }
    public string Type { get; set; }
    public Dictionary<string, object> Properties { get; set; }
}
\end{lstlisting}
\end{frame}

\begin{frame}[fragile]
  \frametitle{Recherche des voisins d’un nœud (extraction de sous-graphe)}
\begin{lstlisting}
public List<Node> GetNeighbors(Node node, List<Relationship> relationships)
{
    return relationships
    .Where(r => r.From == node)
    .Select(r => r.To)
    .ToList();
}
\end{lstlisting}
\end{frame}

\end{document}

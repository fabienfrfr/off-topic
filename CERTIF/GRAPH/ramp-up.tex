\documentclass{beamer}
\usepackage[utf8]{inputenc}
\usepackage[T1]{fontenc}
\usepackage[french]{babel}

\usepackage{amsmath}
\usepackage{amssymb}
\usepackage{listings}
\usepackage{xcolor}
\usepackage{graphicx}
\usepackage{tikz}
\usetikzlibrary{positioning}

% Définition des couleurs utilisées dans TikZ

\definecolor{MediumPurple}{RGB}{147,112,219}
\definecolor{Chartreuse2}{RGB}{102,205,0}
\definecolor{Chartreuse4}{RGB}{69,139,0}

\usetheme{Madrid}
\usecolortheme{seahorse}

% Configuration des listings pour C#, SQL, Python
\lstset{
  language=[Sharp]C,
  basicstyle=\ttfamily\small,
  keywordstyle=\color{blue}\bfseries,
  commentstyle=\color{gray}\itshape,
  stringstyle=\color{red},
  numbers=left,
  numberstyle=\tiny,
  stepnumber=1,
  numbersep=5pt,
  showstringspaces=false,
  breaklines=true,
  frame=single,
  backgroundcolor=\color{gray!10}
}

\title{Knowledge Management System\\Solution Architect}
\author{Fabien FURFARO}
\date{\today}

\begin{document}

\frame{\titlepage}

\begin{frame}
  \frametitle{Sommaire}
  \tableofcontents
\end{frame}

\section{Pourquoi a-t-on besoin de gestionnaire de connaissance ?}

\begin{frame}
  \frametitle{Évolution de la transmission du savoir}
  \begin{itemize}
    \item Transmission orale
    \item Invention de l’écriture
    \item Traduction et diffusion manuscrite
    \item Imprimerie
    \item Révolution industrielle : normalisation
    \item Numérisation : explosion des données, multiplication des supports et des silos
  \end{itemize}
\end{frame}

\section{Comment est géré la connaissance numérique aujourd'hui ?}

\begin{frame}
  \frametitle{Évolution de la transmission du savoir}
  \begin{itemize}
    \item Transmission orale
  \end{itemize}
\end{frame}


\section{Comment fonctionne les bases de données relationnelles ?}

\begin{frame}
  \frametitle{Évolution de la transmission du savoir}
  \begin{itemize}
    \item Transmission orale
  \end{itemize}
\end{frame}

\section{Quels sont les limites des bases de données relationnelles ?}

\begin{frame}
  \frametitle{Comment trouver ma liste d'amis d'amis d'amis et mes preference ?}
  \begin{itemize}
    \item Transmission orale
  \end{itemize}
\end{frame}


\section{Comment les bases de données graphes répondents à cette problématique ?}

\begin{frame}
  \frametitle{Les bases de données graphes, RDF vs Property}
  \begin{itemize}
    \item Transmission orale
  \end{itemize}
\end{frame}


\begin{frame}
  \frametitle{Comparatifs des language de requetes graphes}
  \begin{itemize}
    \item Transmission orale
  \end{itemize}
\end{frame}


\section{Quels autres limites peuvent apparaitres avec les bases de donnée relationnelle ?}

\begin{frame}
  \frametitle{Les base de données vectorielles}
  \begin{itemize}
    \item similarité de texte, mais aussi image/son/video possible!
  \end{itemize}
\end{frame}


\section{Quels sont les architectures modernes pour la gestion de la connaissance ?}

\begin{frame}
  \frametitle{Modulaire vs AllinOne}
  \begin{itemize}
    \item similarité de texte, mais aussi image/son/video possible!
  \end{itemize}
\end{frame}


\end{document}
